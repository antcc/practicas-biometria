
%%% Local Variables:
%%% LaTeX-command: "pdflatex --shell-escape"
%%% End:

\documentclass[11pt]{article}
\usepackage[utf8]{inputenc}
\usepackage[T1]{fontenc}
\usepackage{grffile}
\usepackage{longtable}
\usepackage{wrapfig}
\usepackage{rotating}
\usepackage[normalem]{ulem}
\usepackage{amsmath}
\usepackage{textcomp}
\usepackage{amssymb}
\usepackage{capt-of}
\usepackage{hyperref}
\hypersetup{colorlinks=true, linkcolor=black}
\setlength{\parindent}{0in}
\usepackage[margin=0.8in]{geometry}
\usepackage[english]{babel}
\usepackage{mathtools}
\usepackage{palatino}
\usepackage{fancyhdr}
\usepackage{sectsty}
\usepackage{engord}
\usepackage{parskip}
\usepackage{minted}
\usepackage{cite}
\usepackage{graphicx}
\usepackage{subcaption}
\usepackage{setspace}
\usepackage[compact]{titlesec}
\usepackage[center]{caption}
\usepackage{placeins}
\usepackage{color}
\usepackage{amsmath}
\usepackage{bm}
\usepackage{todonotes}
\usepackage{pdfpages}
% \titlespacing*{\subsection}{0pt}{5.5ex}{3.3ex}
% \titlespacing*{\section}{0pt}{5.5ex}{1ex}
\author{Luis Antonio Ortega Andrés\\Antonio Coín Castro}
\date{}
\title{Fingerprint Biometrics Lab - Report\\\medskip
\large APRENDIZAJE PROFUNDO PARA PROCESAMIENTO DE INFORMACIÓN BIOMÉTRICA}
\hypersetup{
 pdfauthor={Luis Antonio Ortega Andrés},
 pdftitle={},
 pdfkeywords={},
 pdfsubject={},
 pdflang={Spanish}}

\begin{document}

\maketitle

\section*{Exercise 1}
\textbf{a) } \emph{Copy here the two fingerprint images provided as examples (\texttt{example1\_1} and \texttt{example1\_2})}.

\begin{figure}[h!]
  \centering
       \begin{subfigure}[t]{0.45\textwidth}
         \centering
         \includegraphics[scale=1.0]{img/example1_1.png}
         \caption{Example1\_1}
     \end{subfigure}%
     \quad
     \begin{subfigure}[t]{0.45\textwidth}
         \centering
         \includegraphics[scale=1.0]{img/example1_2.png}
         \caption{Example1\_2}
     \end{subfigure}
    \caption{Original fingerprints.}
\end{figure}

\textbf{b) } \emph{How many macro-singularities do you observe in each fingerprint?}

We can see that there is only one macro-singularity in each fingerprint. Specifically, we observe a \textbf{loop} in each one of them.

\textbf{c) } \emph{Mark the macro-singularities in the images (deltas and loops).}

\begin{figure}[h!]
  \centering
       \begin{subfigure}[t]{0.45\textwidth}
         \centering
         \includegraphics[scale=0.7]{img/msing_1.jpg}
         \caption{Example1\_1}
     \end{subfigure}%
     \quad
     \begin{subfigure}[t]{0.45\textwidth}
         \centering
         \includegraphics[scale=0.7]{img/msing_2.jpg}
         \caption{Example1\_2}
     \end{subfigure}
    \caption{Macro-singularities (loops).}
\end{figure}

\section{Exercise 2}

\textbf{a) }\emph{Execute the provided code for Fingerprint Enhancement and paste the resulting image here:}

\begin{figure}[h!]
  \centering
       \begin{subfigure}[t]{0.45\textwidth}
         \centering
         \includegraphics[scale=0.96]{img/enhanced_1}
         \caption{Example1\_1}
     \end{subfigure}%
     \quad
     \begin{subfigure}[t]{0.45\textwidth}
         \centering
         \includegraphics[scale=1.0]{img/enhanced_2}
         \caption{Example1\_2}
     \end{subfigure}
    \caption{Fingerprint Enhancement}
\end{figure}


\textbf{b) }\emph{What differences do you observe with respect to the original fingerprints?}

Fingerprint enhancement facilitate the identification of ridge-valley structures and hence the features of each fingerprint. More precisely, we achieve a representation where any intermittent stroke of a ridge is replaced with a continuous one.

\section{Exercise 3}

\textbf{a) }\emph{Execute now the code for Quality Maps, and past the resulting quality maps:}

\begin{figure}[h!]
  \centering
       \begin{subfigure}[t]{0.45\textwidth}
         \centering
         \includegraphics[scale=1.0]{img/qmap_1}
         \caption{Example1\_1}
     \end{subfigure}%
     \quad
     \begin{subfigure}[t]{0.45\textwidth}
         \centering
         \includegraphics[scale=0.96]{img/qmap_2}
         \caption{Example1\_2}
     \end{subfigure}
    \caption{Fingerprint Enhancement}
\end{figure}

\textbf{b) }\emph{What is the range of values for these quality maps?}

Inspecting each matrix minimum and maximum values, we obtain that for both images the minimum is \( 0 \) and the maximum is \( 0.9991 \) and \(  0.9988 \) for example1 and example2 respectively. Given this, we can assume that the range of values is \( [0, 1] \).

\textbf{c) }\emph{What kind information (apart from the quality) can be inferred from such code?}

Inspecting the given source code, the ``heatmap'' displayed corresponds to a measure of the reliability of the computed orientation of each ridge. In short, this quantity measures the reliability of the computed enhancement of the fingerprint.

Taking a look into our examples, there is one region where the reliability is low (blue), which corresponds to the sharpest point of the loop. This results is intuitive given that those points are the ones where the orientation changes abruptly, leading to less accurate results.


\section{Exercise 4}

\emph{Execute the code in order to show the Binarized Fingerprint and the Segmented Fingerprint. Apply different values of quality threshold (0.1, 0.3, 0.6, 0.9) and paste here the resulting images:}

\begin{figure}
  \centering
  \includegraphics[scale=0.5]{img/merge_bin}
  \caption{Example1\_1 and Example1\_2 (first and second row) Binarized and Segmented Fingerprint with thresholds 0.1, 0.3, 0.6 and 0.9 (columns).}
\end{figure}

\section{Exercise 5}

\textbf{a) }\emph{Execute the code for generating the Fingerprint Skeleton and the Minutiae Extractor. Paste the resulting images for the original values window=5 and margin=5.}

\textbf{b) }\emph{Search heuristically by looking at the images for the optimal values of parameters window and margin. Paste the resulting images with your optimal parameters and justify your decision.}

\section{Exercise 6}

\textbf{a) }\emph{Execute the code corresponding to the Minutiae Validation for window=5 and margin=5.  Paste the resulting image including the minutiae extracted (red crosses) and validated (blue circles) of both fingerprints. }

\textbf{b) }\emph{Execute the same code but with the optimal values of parameters window and margin. Paste the resulting image below.}

\textbf{c) }\emph{Do you think it is a good idea to include the Minutiae Validation module? Justify your opinion.}

\section{Extra Exercise}

\emph{In folder \texttt{/ddbb} you have 20 fingerprint images. 19 of them are labeled with the subject identity (e.g., H0001), and 1 is Unknown. Search for the identity of the Unknown fingerprint in the set of 19 labelled reference fingerprints. You can use the provided code \texttt{identification\_1\_19.m} as basis. Paste here the resulting ranked list of scores of the Unknown fingerprint with respect each one of the 19 reference fingerprints.}

\end{document}
